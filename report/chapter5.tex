\chapter{CÁC GIẢI PHÁP NỔI BẬT VÀ ĐÓNG GÓP}

\section{Giới thiệu}
Chương này trình bày các giải pháp kỹ thuật nổi bật và những đóng góp quan trọng của đề tài trong việc xây dựng hệ thống mạng xã hội âm nhạc. Các giải pháp được đưa ra đều dựa trên kinh nghiệm thực tế trong quá trình phát triển và triển khai hệ thống.

\section{Kiến trúc hệ thống N-Layer truyền thống được tối ưu hóa}

Một trong những đóng góp quan trọng của đề tài là việc triển khai thành công kiến trúc N-Layer truyền thống được tối ưu hóa cho ứng dụng mạng xã hội âm nhạc. Kiến trúc này được thiết kế với bốn tầng chính: Presentation Layer, Business Logic Layer, Data Access Layer và Database Layer.

\subsection{Đặc điểm của kiến trúc được triển khai}

Kiến trúc N-Layer được triển khai trong dự án có những đặc điểm nổi bật sau:

\textbf{Tách biệt rõ ràng các tầng chức năng:} Mỗi tầng có trách nhiệm cụ thể và tương tác với nhau thông qua các interface được định nghĩa rõ ràng. Điều này giúp hệ thống dễ bảo trì và mở rộng.

\textbf{Tầng Presentation được tách biệt hoàn toàn:} Frontend React hoạt động độc lập, giao tiếp với backend thông qua REST API. Điều này cho phép phát triển song song và có thể thay đổi công nghệ frontend mà không ảnh hưởng đến backend.

\textbf{Tầng Business Logic tập trung:} Toàn bộ logic nghiệp vụ được xử lý tại backend Spring Boot, đảm bảo tính nhất quán và bảo mật của dữ liệu.

\textbf{Tầng Data Access được tối ưu:} Sử dụng Spring Data JPA để tự động hóa các thao tác cơ sở dữ liệu, giảm thiểu code boilerplate và tăng hiệu suất.

\subsection{Lợi ích của kiến trúc trong thực tế}

Kiến trúc N-Layer đã mang lại những lợi ích thiết thực trong quá trình phát triển:

\textbf{Khả năng mở rộng cao:} Khi cần thêm tính năng mới, chỉ cần thay đổi tại tầng tương ứng mà không ảnh hưởng đến các tầng khác.

\textbf{Dễ dàng bảo trì:} Mỗi tầng có thể được bảo trì độc lập, giúp việc sửa lỗi và cập nhật trở nên đơn giản hơn.

\textbf{Tái sử dụng code hiệu quả:} Các service và repository có thể được tái sử dụng ở nhiều controller khác nhau.

\textbf{Hiệu suất ổn định:} Thời gian phản hồi API duy trì ở mức 50-200ms, thời gian tải trang từ 1-2 giây trong môi trường thực tế.

\section{Giải pháp xử lý lỗi xác thực người dùng}

\subsection{Vấn đề gặp phải trong thực tế}

Trong quá trình phát triển và kiểm thử hệ thống, một vấn đề nghiêm trọng đã được phát hiện liên quan đến xử lý lỗi xác thực người dùng. Khi người dùng nhập sai thông tin đăng nhập, thay vì hiển thị thông báo lỗi rõ ràng, hệ thống lại tự động tải lại trang đăng nhập, gây ra trải nghiệm người dùng kém và khó hiểu.

Vấn đề này đặc biệt nghiêm trọng vì nó ảnh hưởng trực tiếp đến khả năng sử dụng của hệ thống. Người dùng không thể biết được lý do tại sao việc đăng nhập thất bại, dẫn đến việc họ có thể thử lại nhiều lần hoặc bỏ cuộc sử dụng hệ thống.

\subsection{Phân tích nguyên nhân sâu xa}

Sau khi phân tích kỹ lưỡng, nguyên nhân được xác định là sự kết hợp của hai vấn đề:

\textbf{Thiếu sót trong validation phía backend:} Service xác thực không thực hiện kiểm tra mật khẩu một cách chính xác, dẫn đến việc trả về mã lỗi không phù hợp.

\textbf{Xử lý lỗi không hợp lý ở frontend:} API interceptor được cấu hình để tự động chuyển hướng tất cả các lỗi 401 về trang đăng nhập, mà không phân biệt ngữ cảnh.

Điều này tạo ra một vòng luẩn quẩn: khi đăng nhập sai, backend trả về lỗi 401, frontend nhận được và tự động chuyển hướng về trang đăng nhập, khiến người dùng không nhận được thông báo lỗi.

\subsection{Giải pháp được triển khai}

Để giải quyết vấn đề này, một giải pháp toàn diện đã được triển khai:

\textbf{Cải thiện validation backend:} Thêm logic kiểm tra mật khẩu chính xác trong service xác thực và đảm bảo trả về thông báo lỗi cụ thể cho từng trường hợp.

\textbf{Xử lý exception chuyên biệt:} Tạo ra các exception handler riêng biệt cho các trường hợp lỗi khác nhau, đảm bảo thông báo lỗi được trả về đúng định dạng.

\textbf{Cải tiến API interceptor:} Thay đổi logic của interceptor để chỉ chuyển hướng khi thực sự cần thiết, và cho phép hiển thị thông báo lỗi khi đăng nhập sai.

\textbf{Cải thiện UX frontend:} Hiển thị thông báo lỗi rõ ràng và hữu ích cho người dùng, giúp họ hiểu được vấn đề và biết cách khắc phục.

\subsection{Kết quả đạt được}

Sau khi triển khai giải pháp, hệ thống đã hoạt động ổn định với:

\textbf{Thông báo lỗi chính xác:} Người dùng nhận được thông báo cụ thể khi nhập sai email hoặc mật khẩu.

\textbf{Trải nghiệm người dùng được cải thiện:} Không còn tình trạng trang web tự động tải lại mà không có thông báo.

\textbf{Tăng tỷ lệ thành công trong kiểm thử:} Chức năng xác thực đạt tỷ lệ thành công 100\% trong các test case sau khi khắc phục.

\section{Kết luận}

Chương này đã trình bày hai đóng góp quan trọng của đề tài trong việc xây dựng hệ thống mạng xã hội âm nhạc. 

Việc triển khai thành công kiến trúc N-Layer truyền thống được tối ưu hóa đã tạo ra một nền tảng vững chắc cho hệ thống. Kiến trúc này không chỉ đảm bảo tính ổn định và hiệu suất của ứng dụng mà còn tạo điều kiện thuận lợi cho việc phát triển và bảo trì trong tương lai.

Giải pháp xử lý lỗi xác thực người dùng thể hiện tầm quan trọng của việc phân tích và giải quyết các vấn đề thực tế trong quá trình phát triển phần mềm. Việc khắc phục thành công vấn đề này không chỉ cải thiện trải nghiệm người dùng mà còn nâng cao độ tin cậy của toàn bộ hệ thống.

Những kinh nghiệm và giải pháp được trình bày trong chương này có thể được áp dụng cho các dự án phát triển ứng dụng web tương tự, đặc biệt là các hệ thống có yêu cầu cao về trải nghiệm người dùng và tính ổn định.